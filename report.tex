\documentclass[a4paper,12pt]{article}
\usepackage[T2A]{fontenc}
\usepackage[utf8]{inputenc}
\usepackage[russian]{babel}
\usepackage[style=russian]{csquotes}
\usepackage{xcolor}
\usepackage{amsmath}
\usepackage{tikz}
\usepackage{pgfplots}
\usepackage[l3]{csvsimple}

\usepackage{cleveref}
\crefformat{equation}{(#2#1#3)}
\crefrangeformat{equation}{(#3#1#4)~-~(#5#2#6)}
\crefmultiformat{equation}{(#2#1#3)}{ и~(#2#1#3)}{, (#2#1#3)}{ и~(#2#1#3)}

\allowdisplaybreaks

\patchcmd{\thebibliography}{\section*{Список литературы}}{}{}{}

\newcommand{\UpdateMe}[1]{\textcolor{red}{#1}}
\DeclareMathOperator*{\argmax}{argmax}

\newcommand{\University}{Московский государственный университет имени М.~В.~Ломоносова}
\newcommand{\Department}{Кафедра \UpdateMe{НАЗВАНИЕ-КАФЕДРЫ}}
\newcommand{\Student}{\UpdateMe{ИМЯ-СТУДЕНТА}}
\newcommand{\GroupNum}{\UpdateMe{НОМЕР}}
\newcommand{\Seminar}{Численные методы в задачах оптимального управления}

\begin{document}
% \begin{titlepage}
%     \centering
%     {\scshape\Large \University\par}\vspace{1cm}{\scshape\large \Department\par}
%     \vfill
%     {\huge\bfseries ОТЧЕТ\par}{\Largeпо задаче практикума \enquote{\Seminar}\par}
%     \vfill
%     \hfill\begin{minipage}{0.45\linewidth}Выполнил студент гр. \GroupNum:\\\Student\end{minipage}
%     \vfill
%     {\large Москва, \the\year{}\par}
% \end{titlepage}

\section*{Задание}
Решить задачу оптимального управления ({\bfseries Задача 30}):
\begin{gather*}
  J = \int_0^1 \ddot{x}^2-48\dot{x}/\left(2+\cos{\alpha x}\right)dt \rightarrow \min\\
  x(1)=\dot{x}(0)=0\\
  \alpha\in\{0.0, 0.1, 1.0, 5.1\}
\end{gather*}

\section*{Теория}

Введем обозначения \(x_1=x, x_2=\dot{x}, u=\ddot{x}\). Тогда задача
перепишется в виде:
\begin{gather*}
  \begin{cases}\dot{x}_1=x_2\\ \dot{x}_2=u \end{cases}\\
  x_1(1)=0\\
  x_2(0)=0\\
  \int_0^1 u^2-48x_2/\left(2+\cos{\alpha x_1}\right)dt \rightarrow \min
\end{gather*}

\subsection*{Основные конструкции}

\subsubsection*{Функция Лагранжа}

\begin{gather*}
  \hat{L}=\int_0^1L dt+l\\
  L=\lambda_0 \left(u^2-48x_2/\left(2+\cos{\alpha x_1}\right)\right) + p_1\left(\dot{x}_1-x_2\right)+p_2\left(\dot{x}_2-u\right)=p_1 \dot{x}_1+p_2 \dot{x}_2 - H\\
  l=\lambda_2 x_1(1)+\lambda_3 x_2(0)
\end{gather*}

\subsubsection*{Условия оптимальности}

\begin{enumerate}
\def\labelenumi{\arabic{enumi}.}
\item
  Уравнения
  Эйлера-Лагранжа:
  \begin{equation*}
    \begin{cases}
      -\frac{d}{dt}L_{\dot{x}_1}+L_{x_1}=0\\
      -\frac{d}{dt}L_{\dot{x}_2}+L_{x_2}=0
    \end{cases} \Rightarrow \begin{cases}
      \dot{p}_1=\lambda_0 \frac{-48\alpha x_2\sin{\alpha x_1}}{\left(2+\cos(\alpha x_1)\right)^2}\\
      \dot{p}_2=\lambda_0\frac{-48}{2+\cos{\alpha x_1}}-p_1
    \end{cases}
  \end{equation*}
\item
  Условия трансверсальности:
  \begin{equation*}
    \begin{cases}
      L_{\dot{x}_1}(0)=l_{x_1}(0)\\
      L_{\dot{x}_1}(1)=-l_{x_1}(1)\\
      L_{\dot{x}_2}(0)=l_{x_2}(0)\\
      L_{\dot{x}_2}(1)=-l_{x_2}(1)
    \end{cases} \Rightarrow \begin{cases}
      p_1(0)=0\\
      p_1(1)=-\lambda_2\\
      p_2(0)=\lambda_3\\
      p_2(1)=0
    \end{cases}
  \end{equation*}
\item
  Условие оптимальности по управлению:
  \begin{equation*}
    \tilde{u}=\argmax(H)=\argmax\left(p_2u-\lambda_0 u^2\right) \Rightarrow \tilde{u}=\frac{p_2}{2\lambda_0}, \lambda_0 \neq 0
  \end{equation*}
\item
  Условие невырожденности и неотрицательности:
  \begin{equation*}
    \sum_{i=0}^{3} |\lambda_i| \neq 0, \lambda_0 \ge 0
  \end{equation*}
\end{enumerate}

\subsection*{Постановка задачи}

\subsubsection*{Разбор тривиального случая \((\lambda_0=0)\)}

В этом случае \(\tilde{u}=\argmax(p_2 u)\). При этом возможны 2
ситуации:

\begin{itemize}
\item
  \(p_2 \equiv 0\). Но тогда
  \(\dot{p}_2\equiv 0 \Rightarrow p_1 \equiv 0 \Rightarrow\) Из условий
  трансверсальности \(\forall i \quad \lambda_i=0\) , что противоречит
  условию \textbf{4}.
\item
  \(p_2 \not\equiv 0 \Rightarrow \exists t_0 \in [0, 1] : p_2\left(O_{\varepsilon}(t_0)\right) \neq 0\).
  Тогда при \(t\in O_{\varepsilon}(t_0)\) оптимальное управление
  \(\tilde{u}=\pm\infty \Rightarrow \int_0^1 \tilde{u}^2+...\;dt > \int_{t\in O_\varepsilon(t)}\tilde{u}^2+...\;dt=\infty\),
  что явно не соответствует оптимальному управлению.
\end{itemize}

Таким образом заключаем, что \(\lambda_0 \neq 0\).

\subsubsection*{Общий случай}

В силу однородности функции Лагранжа по множителям в качестве условия
нормировки положим \(\lambda_0=1/2 \Rightarrow u=p_2\). Тогда краевая
задача примет вид:

\begin{equation}
  \label{common_system}
  \begin{cases}
    \dot{x}_1=f_1(x_1,x_2,p_1,p_2)=x_2\\
    \dot{x}_2=f_2(x_1,x_2,p_1,p_2)=p_2 \\
    \dot{p}_1=f_3(x_1,x_2,p_1,p_2)=-\frac{24\alpha x_2\sin{\alpha x_1}}{\left(2+\cos(\alpha x_1\right)^2}\\
    \dot{p}_2=f_4(x_1,x_2,p_1,p_2)=-\frac{24}{2+\cos{\alpha x_1}}-p_1
    \end{cases}
\end{equation}
\begin{equation}
  \label{left_conditions}
  x_2(0)=0, \quad \quad p_1(0)=0
\end{equation}
\begin{equation}
  \label{right_conditions}
  x_1(1)=0, \quad \quad p_2(1)=0
\end{equation}

\section*{Методика решения}

\subsection*{Численный метод решения}

\subsubsection*{Метод Рунге-Кутта}

Пусть дана задача Коши
\begin{equation*}
  \begin{cases} \dot{x}_1=f_1(x_1,x_2,p_1,p_2,t)\\ \dot{x}_2=f_2(x_1,x_2,p_1,p_2,t)\\ \dot{p}_1=f_3(x_1,x_2,p_1,p_2,t)\\ \dot{p}_2=f_4(x_1,x_2,p_1,p_2,t) \end{cases}
\end{equation*}
с известными в начальный момент времени значениями \(x_1^0\), \(\ x_2^0\), \(p_1^0\) и \(p_2^0\).

Допустим уже найдены значения в точке \(t_i\): \(x_1^i\), \(\ x_2^i\), \(p_1^i\) и \(p_2^i\). Тогда значения в точке \(t_{i+1}\) вычисляются по
схеме:

\begin{gather*}
  k_{m1}=h f_m(x_1^i,x_2^i,p_1^i,p_2^i, t) \\ 
  k_{m2}=h f_m(x_1^i + 0.5 k_{11},x_2^i + 0.5 k_{21},p_1^i + 0.5 k_{31},p_2^i + 0.5 k_{41}, t + 0.5 h) \\ 
  \begin{align*}
    k_{m3}=h f_m(&x_1^i + 0.25 (k_{11}+k_{12}),x_2^i + 0.25 (k_{21}+k_{22}),\\
                 &p_1^i + 0.25 (k_{31}+k_{32}),p_2^i + 0.25 (k_{41}+k_{42}), t + 0.5 h)
  \end{align*} \\
  \begin{align*}
    k_{m4}=f_m(&x_1^i - k_{12} + 2 k_{13},x_2^i - k_{22} + 2 k_{23},\\
               &p_1^i - k_{32} + 2 k_{33},p_2^i - k_{42} + 2 k_{43}, t + h)
  \end{align*} \\
  \begin{align*}
    k_{m5}=h f_m(&x_1^i + \frac{1}{27}(7 k_{11}+10 k_{12}+k_{14}),\\
                 &x_2^i + \frac{1}{27}(7 k_{21}+10 k_{22}+k_{24}),\\
                 &p_1^i + \frac{1}{27}(7 k_{31}+10 k_{32}+k_{34}),\\
                 &p_2^i + \frac{1}{27}(7 k_{41}+10 k_{42}+k_{44}), t + \frac{2}{3} h)
  \end{align*} \\
  \begin{align*}
    k_{m6}=h f_m(&x_1^i + \frac{1}{625}(28 k_{11}-125 k_{12}+ 546 k_{13}+54 k_{14}-378 k_{15}),\\
                 &x_2^i + \frac{1}{625}(28 k_{21}-125 k_{22}+ 546 k_{23}+54 k_{24}-378 k_{25}),\\
                 &p_1^i + \frac{1}{625}(28 k_{31}-125 k_{32}+ 546 k_{33}+54 k_{34}-378 k_{35}),\\
                 &p_2^i + \frac{1}{625}(28 k_{41}-125 k_{42}+ 546 k_{43}+54 k_{44}-378 k_{45}), t + \frac{1}{5} h)
  \end{align*} \\
  x_1^{i+1}=\gamma_1^h(x_1^i)=x_1^{i} + \frac{1}{24}k_{11} + \frac{5}{48} k_{14} + \frac{27}{56} k_{15} + \frac{125}{336} k_{16} \\ 
  x_2^{i+1}=\gamma_2^h(x_2^i)=x_2^{i} + \frac{1}{24}k_{21} + \frac{5}{48} k_{24} + \frac{27}{56} k_{25} + \frac{125}{336} k_{26} \\ 
  p_1^{i+1}=\gamma_3^h(p_1^i)=p_1^{i} + \frac{1}{24}k_{31} + \frac{5}{48} k_{34} + \frac{27}{56} k_{35} + \frac{125}{336} k_{36} \\ 
  p_2^{i+1}=\gamma_4^h(p_2^i)=p_2^{i} + \frac{1}{24}k_{41} + \frac{5}{48} k_{44} + \frac{27}{56} k_{45} + \frac{125}{336} k_{46}
\end{gather*}

где \(h=h(t_i)\) - адаптивный шаг по времени.

\subsubsection*{Адаптивный шаг по времени}

Пусть задан уровень погрешности \(\varepsilon\). Обозначим за \(y^i\)
значение искомой функции в текущей точке, а за \(h\) - текущий шаг по
времени.

Погрешность \(m\)-ой функции \(\gamma_{m}\) вычисляется с помощью контрольного члена:
\begin{equation*}
  \delta\gamma_{m} = \frac{-1}{336}\left(42 k_{m1}+224k_{m3}+21k_{m4}-162 k_{m5}-125 k_{m6}\right)
\end{equation*}
Тогда новый шаг на каждой итерации будет вычисляться по формулам
\begin{equation*}
  err = \Vert (\delta\gamma_{1}, \delta\gamma_{2}, \delta\gamma_{3}, \delta\gamma_{4})\Vert
\end{equation*}
\begin{equation*}
  h'=\begin{cases}
    0.95 h \sqrt[6]{\frac{\varepsilon}{x+err}}, \quad err \not\in [0.1 \varepsilon, \varepsilon]\\
    h, \quad  err \in [0.1 \varepsilon, \varepsilon]
  \end{cases}
\end{equation*}
где \(x\) - маленькая
положительная константа избавляющая нас от необходимости проверять, что
\(err \neq 0\). Например можно взять \(x=0.001 \varepsilon\).

\subsubsection*{Метод стрельбы}

Для краевой задачи \cref{common_system} из краевых условий \cref{left_conditions} имеем
\begin{equation*}
  p_1^0=0, \quad x_2^0=0
\end{equation*}

Два других ``начальных'' значения выберем в качестве параметров
пристрелки

\begin{equation*}
  x_1^0=a, \quad p_2^0=b
\end{equation*}

Тогда решая систему \cref{common_system} с такими начальными условиями до момента времени \(t=1\) получим некоторое решение \(\tilde{x}_1\), \(\tilde{x}_2\), \(\tilde{p}_1\) и \(\tilde{p}_2\).

Варьируя параметры \(a\) и \(b\) необходимо добиться выполнения краевых условий \cref{right_conditions}:
\begin{equation*}
  \tilde{x}_1(1)=x_1^N=0, \quad \tilde{p}_2(1)=p_2^N=0
\end{equation*}

Будем искать параметры \(a\) и \(b\) методом простых итераций. Введём
функции \(\phi_1\) и \(\phi_2\) показывающие отклонение от заданного
граничного условия:
\begin{gather*}
  \phi_1(a,b)=x_1^N-0=0\\
  \phi_2(a,b)=p_2^N-0=0
\end{gather*}

Разложив эти функции в окрестности точки \((a^i, b^i)\) получим:

\begin{gather*}
  \phi_1(a,b)\simeq\phi_1(a^i,b^i)+\delta a\frac{\partial \phi_1}{\partial a}(a^i,b^i)+\delta b\frac{\partial \phi_1}{\partial b}(a^i,b^i)\simeq0\\
  \phi_2(a,b)\simeq\phi_2(a^i,b^i)+\delta a\frac{\partial \phi_2}{\partial a}(a^i,b^i)+\delta b\frac{\partial \phi_2}{\partial b}(a^i,b^i)\simeq0
\end{gather*}

где \(\delta z=z^{i+1}-z^{i}\). В матричном виде уравнения примут вид

\begin{equation*}
  \begin{pmatrix} 
    \frac{\partial \phi_1}{\partial a}(a^i,b^i)&\frac{\partial \phi_1}{\partial b}(a^i,b^i)\\
    \frac{\partial \phi_2}{\partial a}(a^i,b^i)&\frac{\partial \phi_2}{\partial b}(a^i,b^i)
  \end{pmatrix} \cdot \begin{pmatrix}
    \delta a\\
    \delta b
  \end{pmatrix} =W\cdot\begin{pmatrix}
    \delta a\\
    \delta b
  \end{pmatrix} = -\begin{pmatrix}
    \phi_1(a^i,b^i)\\
    \phi_2(a^i,b^i)
  \end{pmatrix}
\end{equation*}

Тогда итерационный процесс может быть выражен формулой

\begin{equation*}
  \begin{pmatrix}
    a^{i+1}\\
    b^{i+1}
  \end{pmatrix}=\begin{pmatrix}
    a^{i}\\
    b^{i}
  \end{pmatrix}-W^{-1}\cdot \begin{pmatrix}
    \phi_1(a^i,b^i)\\
    \phi_2(a^i,b^i)
  \end{pmatrix}
\end{equation*}

\subsection*{Аналитическое решение при
\(\alpha=0\)}

В данном случае система \cref{common_system} разбивается на две подсистемы:

\begin{equation}
  \label{simple_x}
  \begin{cases}
    \dot{p}_1=0\\
    \dot{p}_2=-8-p_1
  \end{cases}
\end{equation}

\begin{equation}
  \label{simple_p}
  \begin{cases}
    \dot{x}_1=x_2\\
    \dot{x}_2=p_2
  \end{cases}
\end{equation}

Из уравнения \cref{simple_p} с учетом граничных условий \cref{left_conditions,right_conditions} получим

\(p_1=0, \quad p_2=8\left(1-t\right)\)

Тогда из уравнения \cref{simple_x} найдём выражения для \(x_1\) и \(x_2\)

\(x_2=4\left(1-t\right)^2+C_2\)

\(x_1=\frac{4}{3}\left(1-t\right)^3+C_2 t+C_3\)

Подставив эти функции в оставшиеся граничные условия \cref{left_conditions,right_conditions} получим систему на коэффициенты

\begin{equation*}
  \begin{cases}
    4+C_2=0\\
    C_2+C_3=0
  \end{cases} \Rightarrow x=\frac{4}{3}\left(1-t\right)^3-4 (1-t)
\end{equation*}

\section*{Результаты}
Начальные параметры \(x_1^0\) и \(p_2^0\) искались итеративно для последовательных \(\alpha\): \(\left(x_1^0, p_2^0\right)\) для \(\alpha=0.0\) принимались за стартовую точку для \(\alpha=0.1\) и т.д. 

В случае \(\det\vert W\vert=0\) значения сбрасывались на случайные из интервала \([-10, 10]\).

\subsection*{Случай \(\alpha=0.0\)}
\begin{tikzpicture}
  \begin{axis}[width=15cm, height=8cm, legend pos=outer north east, grid = major, grid style={dashed, gray!30}]
    \addplot [thick, mark = none, red]table [x=t, y=x1, col sep=comma] {data/points_alpha0.0.csv};
    \addlegendentry{$x_1$}
    \addplot [only marks, mark = x, black]table [x=t, y=x1, col sep=comma] {data/analitic_points_alpha0.0.csv};
    \addlegendentry{$x_1^t$}
    \addplot [thick, mark = none, blue]table [x=t, y=x2, col sep=comma] {data/points_alpha0.0.csv};
    \addlegendentry{$x_2$}
    \addplot [thick, mark = none, green]table [x=t, y=p1, col sep=comma] {data/points_alpha0.0.csv};
    \addlegendentry{$p_1$}
    \addplot [thick, mark = none, black]table [x=t, y=p2, col sep=comma] {data/points_alpha0.0.csv};
    \addlegendentry{$p_2$}
  \end{axis}
\end{tikzpicture}

Как видно их графика, аналитическое решение \(x_1^t\) и численное \(x_1\) совпадают.

\subsection*{Случай \(\alpha=0.1\)}
\begin{tikzpicture}
  \begin{axis}[width=15cm, height=8cm, legend pos=outer north east, grid = major, grid style={dashed, gray!30}]
    \addplot [thick, mark = none, red]table [x=t, y=x1, col sep=comma] {data/points_alpha0.1.csv};
    \addlegendentry{$x_1$}
    \addplot [thick, mark = none, blue]table [x=t, y=x2, col sep=comma] {data/points_alpha0.1.csv};
    \addlegendentry{$x_2$}
    \addplot [thick, mark = none, green]table [x=t, y=p1, col sep=comma] {data/points_alpha0.1.csv};
    \addlegendentry{$p_1$}
    \addplot [thick, mark = none, black]table [x=t, y=p2, col sep=comma] {data/points_alpha0.1.csv};
    \addlegendentry{$p_2$}
  \end{axis}
\end{tikzpicture}

\subsection*{Случай \(\alpha=1.0\)}
\begin{tikzpicture}
  \begin{axis}[width=15cm, height=8cm, legend pos=outer north east, grid = major, grid style={dashed, gray!30}]
    \addplot [thick, mark = none, red]table [x=t, y=x1, col sep=comma] {data/points_alpha1.0.csv};
    \addlegendentry{$x_1$}
    \addplot [thick, mark = none, blue]table [x=t, y=x2, col sep=comma] {data/points_alpha1.0.csv};
    \addlegendentry{$x_2$}
    \addplot [thick, mark = none, green]table [x=t, y=p1, col sep=comma] {data/points_alpha1.0.csv};
    \addlegendentry{$p_1$}
    \addplot [thick, mark = none, black]table [x=t, y=p2, col sep=comma] {data/points_alpha1.0.csv};
    \addlegendentry{$p_2$}
  \end{axis}
\end{tikzpicture}

\subsection*{Случай \(\alpha=5.1\)}
\begin{tikzpicture}
  \begin{axis}[width=15cm, height=8cm, legend pos=outer north east, grid = major, grid style={dashed, gray!30}]
    \addplot [thick, mark = none, red]table [x=t, y=x1, col sep=comma] {data/points_alpha5.1.csv};
    \addlegendentry{$x_1$}
    \addplot [thick, mark = none, blue]table [x=t, y=x2, col sep=comma] {data/points_alpha5.1.csv};
    \addlegendentry{$x_2$}
    \addplot [thick, mark = none, green]table [x=t, y=p1, col sep=comma] {data/points_alpha5.1.csv};
    \addlegendentry{$p_1$}
    \addplot [thick, mark = none, black]table [x=t, y=p2, col sep=comma] {data/points_alpha5.1.csv};
    \addlegendentry{$p_2$}
  \end{axis}
\end{tikzpicture}

\subsection*{Сводная таблица}
\csvautotabular{data/stats.csv}


\addcontentsline{toc}{section}{Список литературы}
\begin{thebibliography}{99}
  \bibitem{RKM}\textbf{О.Б. Арушанян, С.Ф. Залеткин} Решение систем обыкновенных дифференциальных уравнений методами Рунге--Кутта.
  \bibitem{TAC}\textbf{Д.П. Ким} Теория автоматического управления. Том 2.
  \bibitem{OM}\textbf{Н.Л. Майорова, Д.В. Глазков} Методы оптимизации.
  \bibitem{DEP}\textbf{А.Ф. Филиппов} Сборник задач по дифференциальным уравнениям.
\end{thebibliography}
\end{document}