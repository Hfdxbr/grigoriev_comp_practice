\documentclass[a4paper,12pt]{article}
\usepackage[T2A]{fontenc}
\usepackage[utf8]{inputenc}
\usepackage[russian]{babel}
\usepackage[style=russian]{csquotes}
\usepackage{xcolor}
\usepackage{amsmath}
\usepackage{tikz}
\usepackage{pgfplots}
\usepackage[l3]{csvsimple}

\usepackage{cleveref}
\crefformat{equation}{(#2#1#3)}
\crefrangeformat{equation}{(#3#1#4)~-~(#5#2#6)}
\crefmultiformat{equation}{(#2#1#3)}{ и~(#2#1#3)}{, (#2#1#3)}{ и~(#2#1#3)}

\allowdisplaybreaks

\patchcmd{\thebibliography}{\section*{Список литературы}}{}{}{}

\newcommand{\UpdateMe}[1]{\textcolor{red}{#1}}
\DeclareMathOperator*{\argmax}{argmax}

\newcommand{\University}{Московский государственный университет имени М.~В.~Ломоносова}
\newcommand{\Department}{Кафедра \UpdateMe{НАЗВАНИЕ-КАФЕДРЫ}}
\newcommand{\Student}{\UpdateMe{ИМЯ-СТУДЕНТА}}
\newcommand{\GroupNum}{\UpdateMe{НОМЕР}}
\newcommand{\Seminar}{Численные методы в задачах оптимального управления}

\begin{document}
% \begin{titlepage}
%     \centering
%     {\scshape\Large \University\par}\vspace{1cm}{\scshape\large \Department\par}
%     \vfill
%     {\huge\bfseries ОТЧЕТ\par}{\Largeпо задаче практикума \enquote{\Seminar}\par}
%     \vfill
%     \hfill\begin{minipage}{0.45\linewidth}Выполнил студент гр. \GroupNum:\\\Student\end{minipage}
%     \vfill
%     {\large Москва, \the\year{}\par}
% \end{titlepage}

\section*{Задание}
Решить задачу оптимального управления ({\bfseries Задача 33}):
\begin{gather*}
  J = \int_0^1 \ddot{x}^2/\left(1+\alpha t^4\right)dt \rightarrow \min\\
  x(1)=\dot{x}(1)=0,\quad \dot{x}(0) = 1\\
  \int_0^1xdt=1\\
  \alpha\in\{0.0, 0.01, 0.5, 1.5, 10.5\}
\end{gather*}

\section*{Теория}

Введем обозначения \(x_1=x, x_2=\dot{x}, u=\ddot{x}\). Тогда задача
перепишется в виде:
\begin{gather*}
  \begin{cases}\dot{x}_1=x_2\\ \dot{x}_2=u \end{cases}\\
  x_1(0)=0\\
  x_2(0)=1,\quad x_2(1)=0\\
  \int_0^1 u^2/\left(1+\alpha t^4\right)dt \rightarrow \min\\
  \int_0^1 x_1 dt=1
\end{gather*}

\subsection*{Основные конструкции}

\subsubsection*{Функция Лагранжа}

\begin{gather*}
  \hat{L}=\int_0^1L dt+l\\
  L=\lambda_0 \frac{u^2}{1+\alpha t^4} + \lambda_1 x_1+p_1\left(\dot{x}_1-x_2\right)+p_2\left(\dot{x}_2-u\right)=p_1 \dot{x}_1+p_2 \dot{x}_2 - H\\
  l=\lambda_2 x_1(0)+\lambda_3 x_2(0)+\lambda_4 x_2(1)
\end{gather*}

\subsubsection*{Условия оптимальности}

\begin{enumerate}
\def\labelenumi{\arabic{enumi}.}
\item
  Уравнения
  Эйлера-Лагранжа:
  \begin{equation*}
    \begin{cases}
      -\frac{d}{dt}L_{\dot{x}_1}+L_{x_1}=0\\
      -\frac{d}{dt}L_{\dot{x}_2}+L_{x_2}=0
    \end{cases} \Rightarrow \begin{cases}
      \dot{p}_1=\lambda_1\\
      \dot{p}_2=-p_1
    \end{cases}
  \end{equation*}
\item
  Условия трансверсальности:
  \begin{equation*}
    \begin{cases}
      L_{\dot{x}_1}(0)=l_{x_1}(0)\\
      L_{\dot{x}_1}(1)=-l_{x_1}(1)\\
      L_{\dot{x}_2}(0)=l_{x_2}(0)\\
      L_{\dot{x}_2}(1)=-l_{x_2}(1)
    \end{cases} \Rightarrow \begin{cases}
      p_1(0)=\lambda_2\\
      p_1(1)=0\\
      p_2(0)=\lambda_3\\
      p_2(1)=-\lambda_4
    \end{cases}
  \end{equation*}
\item
  Условие оптимальности по управлению:
  \begin{equation*}
    \tilde{u}=\argmax(H)=\argmax\left(p_2u-\lambda_0 \frac{u^2}{1+\alpha t^4}\right)
    \Rightarrow
    \tilde{u}=p_2\frac{1+\alpha t^4}{2\lambda_0}, \lambda_0 \neq0
  \end{equation*}
\item
  Условие невырожденности и неотрицательности:
  \begin{equation*}
    \sum_{i=0}^{4} |\lambda_i| \neq 0, \lambda_0 \ge 0
  \end{equation*}
\end{enumerate}

\subsection*{Постановка задачи}

\subsubsection*{Разбор тривиального случая \((\lambda_0=0)\)}

В этом случае \(\tilde{u}=\argmax(p_2 u)\). При этом возможны 2
ситуации:

\begin{itemize}
\item
  \(p_2 \equiv 0\). Но тогда
  \(\dot{p}_2\equiv 0 \Rightarrow p_1 \equiv 0 \Rightarrow\) Из условий
  трансверсальности \(\forall i \quad \lambda_i=0\) , что противоречит
  условию \textbf{4}.
\item
  \(p_2 \not\equiv 0 \Rightarrow \exists t_0 \in [0, 1] : p_2\left(O_{\varepsilon}(t_0)\right) \neq 0\).
  Тогда при \(t\in O_{\varepsilon}(t_0)\) оптимальное управление
  \(\tilde{u}=\pm\infty \Rightarrow \int_0^1 \frac{\tilde{u}^2}{1+\alpha t^4}dt > \int_{t\in O_\varepsilon(t)}\frac{\tilde{u}^2}{1+\alpha t^4}dt=\infty\),
  что явно не соответствует оптимальному управлению.
\end{itemize}

Таким образом заключаем, что \(\lambda_0 \neq 0\).

\subsubsection*{Общий случай}

В силу однородности функции Лагранжа по множителям в качестве условия
нормировки положим \(\lambda_0=1/2 \Rightarrow u=p_2\). Тогда краевая
задача примет вид:

\begin{equation}
  \label{common_system}
  \begin{cases}
    \dot{x}_1=f_1(x_1,x_2,p_1,p_2)=x_2\\
    \dot{x}_2=f_2(x_1,x_2,p_1,p_2)=p_2\left(1+\alpha t^4\right) \\
    \dot{p}_1=f_3(x_1,x_2,p_1,p_2)=\lambda_1\\
    \dot{p}_2=f_4(x_1,x_2,p_1,p_2)=-p_1
    \end{cases}
\end{equation}
\begin{equation}
  \label{left_conditions}
  x_1(0)=0, \quad x_2(0)=1
\end{equation}
\begin{equation}
  \label{right_conditions}
  x_2(1)=0, \quad p_1(1)=0
\end{equation}
\begin{equation}
  \label{integral_condition}
  \int_0^1x_1dt=1
\end{equation}

\section*{Методика решения}

\subsection*{Численный метод решения}

\subsubsection*{Метод Рунге-Кутта}

Пусть дана задача Коши
\begin{equation*}
  \begin{cases}
    \dot{x}_1=f_1(x_1,x_2,p_1,p_2,t)\\
    \dot{x}_2=f_2(x_1,x_2,p_1,p_2,t)\\
    \dot{p}_1=f_3(x_1,x_2,p_1,p_2,t)\\
    \dot{p}_2=f_4(x_1,x_2,p_1,p_2,t)
  \end{cases}
\end{equation*}
с известными в начальный момент времени значениями \(x_1^0\), \(\ x_2^0\), \(p_1^0\) и \(p_2^0\).

Допустим уже найдены значения в точке \(t_i\): \(x_1^i\), \(\ x_2^i\), \(p_1^i\) и \(p_2^i\). Тогда значения в точке \(t_{i+1}\) вычисляются по
схеме:

\begin{gather*}
  k_{m1}=h f_m(x_1^i,x_2^i,p_1^i,p_2^i, t) \\ 
  k_{m2}=h f_m(x_1^i + 0.5 k_{11},x_2^i + 0.5 k_{21},p_1^i + 0.5 k_{31},p_2^i + 0.5 k_{41}, t + 0.5 h) \\ 
  \begin{align*}
    k_{m3}=h f_m(&x_1^i + 0.25 (k_{11}+k_{12}),x_2^i + 0.25 (k_{21}+k_{22}),\\
                 &p_1^i + 0.25 (k_{31}+k_{32}),p_2^i + 0.25 (k_{41}+k_{42}), t + 0.5 h)
  \end{align*} \\
  \begin{align*}
    k_{m4}=f_m(&x_1^i - k_{12} + 2 k_{13},x_2^i - k_{22} + 2 k_{23},\\
               &p_1^i - k_{32} + 2 k_{33},p_2^i - k_{42} + 2 k_{43}, t + h)
  \end{align*} \\
  \begin{align*}
    k_{m5}=h f_m(&x_1^i + \frac{1}{27}(7 k_{11}+10 k_{12}+k_{14}),\\
                 &x_2^i + \frac{1}{27}(7 k_{21}+10 k_{22}+k_{24}),\\
                 &p_1^i + \frac{1}{27}(7 k_{31}+10 k_{32}+k_{34}),\\
                 &p_2^i + \frac{1}{27}(7 k_{41}+10 k_{42}+k_{44}), t + \frac{2}{3} h)
  \end{align*} \\
  \begin{align*}
    k_{m6}=h f_m(&x_1^i + \frac{1}{625}(28 k_{11}-125 k_{12}+ 546 k_{13}+54 k_{14}-378 k_{15}),\\
                 &x_2^i + \frac{1}{625}(28 k_{21}-125 k_{22}+ 546 k_{23}+54 k_{24}-378 k_{25}),\\
                 &p_1^i + \frac{1}{625}(28 k_{31}-125 k_{32}+ 546 k_{33}+54 k_{34}-378 k_{35}),\\
                 &p_2^i + \frac{1}{625}(28 k_{41}-125 k_{42}+ 546 k_{43}+54 k_{44}-378 k_{45}), t + \frac{1}{5} h)
  \end{align*} \\
  x_1^{i+1}=\gamma_1^h(x_1^i)=x_1^{i} + \frac{1}{24}k_{11} + \frac{5}{48} k_{14} + \frac{27}{56} k_{15} + \frac{125}{336} k_{16} \\ 
  x_2^{i+1}=\gamma_2^h(x_2^i)=x_2^{i} + \frac{1}{24}k_{21} + \frac{5}{48} k_{24} + \frac{27}{56} k_{25} + \frac{125}{336} k_{26} \\ 
  p_1^{i+1}=\gamma_3^h(p_1^i)=p_1^{i} + \frac{1}{24}k_{31} + \frac{5}{48} k_{34} + \frac{27}{56} k_{35} + \frac{125}{336} k_{36} \\ 
  p_2^{i+1}=\gamma_4^h(p_2^i)=p_2^{i} + \frac{1}{24}k_{41} + \frac{5}{48} k_{44} + \frac{27}{56} k_{45} + \frac{125}{336} k_{46}
\end{gather*}

где \(h=h(t_i)\) - адаптивный шаг по времени.

\subsubsection*{Адаптивный шаг по времени}

Пусть задан уровень погрешности \(\varepsilon\). Обозначим за \(y^i\)
значение искомой функции в текущей точке, а за \(h\) - текущий шаг по
времени.

Погрешность \(m\)-ой функции \(\gamma_{m}\) вычисляется с помощью контрольного члена:
\begin{equation*}
  \delta\gamma_{m} = \frac{-1}{336}\left(42 k_{m1}+224k_{m3}+21k_{m4}-162 k_{m5}-125 k_{m6}\right)
\end{equation*}
Тогда новый шаг на каждой итерации будет вычисляться по формулам
\begin{equation*}
  err = \Vert (\delta\gamma_{1}, \delta\gamma_{2}, \delta\gamma_{3}, \delta\gamma_{4})\Vert
\end{equation*}
\begin{equation*}
  h'=\begin{cases}
    0.95 h \sqrt[6]{\frac{\varepsilon}{x+err}}, \quad err \not\in [0.1 \varepsilon, \varepsilon]\\
    h, \quad  err \in [0.1 \varepsilon, \varepsilon]
  \end{cases}
\end{equation*}
где \(x\) - маленькая
положительная константа избавляющая нас от необходимости проверять, что
\(err \neq 0\). Например можно взять \(x=0.001 \varepsilon\).

\subsubsection*{Метод стрельбы}

Для краевой задачи \cref{common_system} из краевых условий \cref{left_conditions} имеем
\begin{equation*}
  x_1^0=0, \quad x_2^0=1
\end{equation*}

Два других ``начальных'' значения выберем в качестве параметров
пристрелки

\begin{equation*}
  p_1^0=a, \quad p_2^0=b, \quad \lambda_1 =c
\end{equation*}

Тогда решая систему \cref{common_system} с такими начальными условиями до момента времени \(t=1\) получим некоторое решение \(\tilde{x}_1\), \(\tilde{x}_2\), \(\tilde{p}_1\) и \(\tilde{p}_2\).

Варьируя параметры \(a\), \(b\) и \(c\) необходимо добиться выполнения краевых условий \cref{right_conditions} и интегрального условия \cref{integral_condition}:
\begin{gather*}
  \tilde{x}_2(1)=x_2^N=0, \quad \tilde{p}_1(1)=p_1^N=0\\
  \int_0^1\tilde{x}_1dt=S(x_1)=\frac{x_1^0}{2} (t_1-t_0) + \frac{x_1^i}{2}(t_{i+1}-t_{i-1}) + \frac{x_1^N}{2} (t_N-t_{N-1})=1
\end{gather*}

Будем искать параметры \(a\), \(b\) и \(c\) методом простых итераций. Введём
функции \(\phi_1\), \(\phi_2\) и \(\phi_3\) показывающие отклонение от заданного
граничного условия:
\begin{gather*}
  \phi_1(a,b,c)=x_2^N-0=0\\
  \phi_2(a,b,c)=p_1^N-0=0\\
  \phi_3(a,b,c)=S(x_1)-1=0
\end{gather*}

Разложив эти функции в окрестности точки \((a^i, b^i, c^i)\) получим:

\begin{gather*}
  \phi_1(a,b,c)\simeq\phi_1(a^i,b^i,c^i)+\delta a\frac{\partial \phi_1}{\partial a}(a^i,b^i,c^i)+\delta b\frac{\partial \phi_1}{\partial b}(a^i,b^i,c^i)+\delta c\frac{\partial \phi_1}{\partial c}(a^i,b^i,c^i)\simeq0\\
  \phi_2(a,b,c)\simeq\phi_2(a^i,b^i,c^i)+\delta a\frac{\partial \phi_2}{\partial a}(a^i,b^i,c^i)+\delta b\frac{\partial \phi_2}{\partial b}(a^i,b^i,c^i)+\delta c\frac{\partial \phi_2}{\partial c}(a^i,b^i,c^i)\simeq0\\
  \phi_3(a,b,c)\simeq\phi_3(a^i,b^i,c^i)+\delta a\frac{\partial \phi_3}{\partial a}(a^i,b^i,c^i)+\delta b\frac{\partial \phi_3}{\partial b}(a^i,b^i,c^i)+\delta c\frac{\partial \phi_3}{\partial c}(a^i,b^i,c^i)\simeq0
\end{gather*}

где \(\delta z=z^{i+1}-z^{i}\). В матричном виде уравнения примут вид

\begin{equation*}
  \begin{pmatrix} 
    \frac{\partial \phi_1}{\partial a}(a^i,b^i,c^i)&\frac{\partial \phi_1}{\partial b}(a^i,b^i,c^i)&\frac{\partial \phi_1}{\partial c}(a^i,b^i,c^i)\\
    \frac{\partial \phi_2}{\partial a}(a^i,b^i,c^i)&\frac{\partial \phi_2}{\partial b}(a^i,b^i,c^i)&\frac{\partial \phi_2}{\partial c}(a^i,b^i,c^i)\\
    \frac{\partial \phi_3}{\partial a}(a^i,b^i,c^i)&\frac{\partial \phi_3}{\partial b}(a^i,b^i,c^i)&\frac{\partial \phi_3}{\partial c}(a^i,b^i,c^i)\\
  \end{pmatrix} \cdot \begin{pmatrix}
    \delta a\\
    \delta b\\
    \delta c
  \end{pmatrix} =W\cdot\begin{pmatrix}
    \delta a\\
    \delta b\\
    \delta c
  \end{pmatrix} = -\begin{pmatrix}
    \phi_1(a^i,b^i,c^i)\\
    \phi_2(a^i,b^i,c^i)\\
    \phi_3(a^i,b^i,c^i)
  \end{pmatrix}
\end{equation*}

Тогда итерационный процесс может быть выражен формулой

\begin{equation*}
  \begin{pmatrix}
    a^{i+1}\\
    b^{i+1}\\
    c^{i+1}
  \end{pmatrix}=\begin{pmatrix}
    a^{i}\\
    b^{i}\\
    c^{i}
  \end{pmatrix}-W^{-1}\cdot \begin{pmatrix}
    \phi_1(a^i,b^i,c^i)\\
    \phi_2(a^i,b^i,c^i)\\
    \phi_3(a^i,b^i,c^i)
  \end{pmatrix}
\end{equation*}

\subsection*{Аналитическое решение при
\(\alpha=0\)}

В данном случае система \cref{common_system} разбивается на две подсистемы:

\begin{equation}
  \label{simple_p}
  \begin{cases}
    \dot{p}_1=\lambda_1\\
    \dot{p}_2=-p_1
  \end{cases}
\end{equation}

\begin{equation}
  \label{simple_x}
  \begin{cases}
    \dot{x}_1=x_2\\
    \dot{x}_2=p_2
  \end{cases}
\end{equation}

Из уравнения \cref{simple_p} с учетом граничных условий \cref{left_conditions,right_conditions} получим

\(p_1=\lambda_1(t-1), \quad p_2=-\lambda_1\left(t^2/2-t\right)+C_1\)

Тогда из уравнения \cref{simple_x} найдём выражения для \(x_1\) и \(x_2\)

\(x_2=-\lambda_1\left(t^3/6-t^2/2\right)+C_1 t +C_2\)

\(x_1=-\lambda_1\left(t^4/24-t^3/6\right)+C_1 t^2/2 +C_2 t+ C_3\)

Подставив эти функции в оставшиеся граничные условия \cref{left_conditions,right_conditions, integral_condition} получим систему на коэффициенты

\begin{equation*}
  \begin{cases}
    C_3=0\\
    C_2=1\\
    C_1+C_2+\lambda_1/3=0\\
    C_1/6+C_2/2+C_3+\lambda_1/30=1
  \end{cases}
  \Rightarrow
  \begin{cases}
    \lambda_1=-30\\
    C_1=9\\
    C_2=1\\
    C_3=0
  \end{cases}
\end{equation*}

\section*{Результаты}
Начальные параметры \(p_1^0\), \(p_2^0\) и \(\lambda_1\) искались итеративно для последовательных \(\alpha\): \(\left(p_1^0, p_2^0, \lambda_1\right)\) для \(\alpha=0.0\) принимались за стартовую точку для \(\alpha=0.01\) и т.д. 

В случае \(\det\vert W\vert=0\) значения сбрасывались на случайные из интервала \([-10, 10]\).

\subsection*{Случай \(\alpha=0.0\)}
\begin{tikzpicture}
  \begin{axis}[width=15cm, height=8cm, legend pos=outer north east, grid = major, grid style={dashed, gray!30}]
    \addplot [thick, mark = none, red]table [x=t, y=x1, col sep=comma] {data/points_alpha0.0.csv};
    \addlegendentry{$x_1$}
    \addplot [only marks, mark = x, black]table [x=t, y=x1, col sep=comma] {data/analitic_points_alpha0.0.csv};
    \addlegendentry{$x_1^t$}
    \addplot [thick, mark = none, blue]table [x=t, y=x2, col sep=comma] {data/points_alpha0.0.csv};
    \addlegendentry{$x_2$}
    \addplot [thick, mark = none, green]table [x=t, y=p1, col sep=comma] {data/points_alpha0.0.csv};
    \addlegendentry{$p_1$}
    \addplot [thick, mark = none, black]table [x=t, y=p2, col sep=comma] {data/points_alpha0.0.csv};
    \addlegendentry{$p_2$}
  \end{axis}
\end{tikzpicture}

Как видно их графика, аналитическое решение \(x_1^t\) и численное \(x_1\) совпадают.

\subsection*{Случай \(\alpha=0.01\)}
\begin{tikzpicture}
  \begin{axis}[width=15cm, height=8cm, legend pos=outer north east, grid = major, grid style={dashed, gray!30}]
    \addplot [thick, mark = none, red]table [x=t, y=x1, col sep=comma] {data/points_alpha0.01.csv};
    \addlegendentry{$x_1$}
    \addplot [thick, mark = none, blue]table [x=t, y=x2, col sep=comma] {data/points_alpha0.01.csv};
    \addlegendentry{$x_2$}
    \addplot [thick, mark = none, green]table [x=t, y=p1, col sep=comma] {data/points_alpha0.01.csv};
    \addlegendentry{$p_1$}
    \addplot [thick, mark = none, black]table [x=t, y=p2, col sep=comma] {data/points_alpha0.01.csv};
    \addlegendentry{$p_2$}
  \end{axis}
\end{tikzpicture}

\subsection*{Случай \(\alpha=0.5\)}
\begin{tikzpicture}
  \begin{axis}[width=15cm, height=8cm, legend pos=outer north east, grid = major, grid style={dashed, gray!30}]
    \addplot [thick, mark = none, red]table [x=t, y=x1, col sep=comma] {data/points_alpha0.5.csv};
    \addlegendentry{$x_1$}
    \addplot [thick, mark = none, blue]table [x=t, y=x2, col sep=comma] {data/points_alpha0.5.csv};
    \addlegendentry{$x_2$}
    \addplot [thick, mark = none, green]table [x=t, y=p1, col sep=comma] {data/points_alpha0.5.csv};
    \addlegendentry{$p_1$}
    \addplot [thick, mark = none, black]table [x=t, y=p2, col sep=comma] {data/points_alpha0.5.csv};
    \addlegendentry{$p_2$}
  \end{axis}
\end{tikzpicture}

\subsection*{Случай \(\alpha=1.5\)}
\begin{tikzpicture}
  \begin{axis}[width=15cm, height=8cm, legend pos=outer north east, grid = major, grid style={dashed, gray!30}]
    \addplot [thick, mark = none, red]table [x=t, y=x1, col sep=comma] {data/points_alpha1.5.csv};
    \addlegendentry{$x_1$}
    \addplot [thick, mark = none, blue]table [x=t, y=x2, col sep=comma] {data/points_alpha1.5.csv};
    \addlegendentry{$x_2$}
    \addplot [thick, mark = none, green]table [x=t, y=p1, col sep=comma] {data/points_alpha1.5.csv};
    \addlegendentry{$p_1$}
    \addplot [thick, mark = none, black]table [x=t, y=p2, col sep=comma] {data/points_alpha1.5.csv};
    \addlegendentry{$p_2$}
  \end{axis}
\end{tikzpicture}

\subsection*{Случай \(\alpha=10.5\)}
\begin{tikzpicture}
  \begin{axis}[width=15cm, height=8cm, legend pos=outer north east, grid = major, grid style={dashed, gray!30}]
    \addplot [thick, mark = none, red]table [x=t, y=x1, col sep=comma] {data/points_alpha10.5.csv};
    \addlegendentry{$x_1$}
    \addplot [thick, mark = none, blue]table [x=t, y=x2, col sep=comma] {data/points_alpha10.5.csv};
    \addlegendentry{$x_2$}
    \addplot [thick, mark = none, green]table [x=t, y=p1, col sep=comma] {data/points_alpha10.5.csv};
    \addlegendentry{$p_1$}
    \addplot [thick, mark = none, black]table [x=t, y=p2, col sep=comma] {data/points_alpha10.5.csv};
    \addlegendentry{$p_2$}
  \end{axis}
\end{tikzpicture}

\subsection*{Сводная таблица}
\csvautotabular{data/stats.csv}


\section*{Анализ оптимальности}
Для исследования оптимальности найденной экстремали проверим условия второго порядка:
\begin{itemize}
  \item \textbf{Условие Лежандра}. Так как \(L_{uu}=2/\left(1+\alpha t^4\right) > 0\), то выполняется усиленное условие Лежандра, и необходимое условие минимума выполнено.
  \item \textbf{Условие Якоби}. Точка \(\tau\) является сопряжённой, если существует нетривиальное решение следующей краевой задачи:
  \begin{gather*}
    \begin{cases}
      \dot{\delta x}_1=\delta x_2\\
      \dot{\delta x}_2=q_2\left(1+\alpha t^4\right) \\
      \dot{q}_1=0\\
      \dot{q}_2=-q_1
    \end{cases}\\
    \delta x_1(\tau)=\delta x_2(\tau)=0\\
    \delta x_1(0) = \delta x_2(0) = \delta x_2(1) = 0\\
    q_1(1)=0
  \end{gather*}
  Решая систему аналитически, получаем, что существует только тривиальное решение. Таким образом сопряжённых точек на полуотрезке \((0,1]\) и выполняется усиленное условие Якоби.
  \item \textbf{Условие Вейерштрасса}. Так как \(L_{uu}=2/\left(1+\alpha t^4\right)\) положительно определена, то выполняется условие квазирегулярности (усиленное условие Вейерштрасса).
  \item \textbf{Исследование второй вариации функции Лагранжа}. Функция Лагранжа и её вторая вариация для данной задачи имеют вид:
  \begin{equation*}
    \hat{L}=\int_0^1L dt+l, \hat{L}_{\xi\xi}=\int_0^1(\delta u)^2 dt
  \end{equation*}
  Достаточным условием второго порядка является положительная определённость \(\hat{L}_{\xi\xi}\) на допустимых вариациях:
  \begin{equation*}
    \hat{L}_{\xi\xi}(\delta\xi,\delta\xi) \ge \varepsilon \phi(\delta\xi), \quad \varepsilon > 0
  \end{equation*}
  В качестве \(\phi(\delta\xi)\) можно взять саму функцию \(\hat{L}\). Условие выполняется.
\end{itemize}
Так как задача линейно квадратичная, в \(\tilde{x}\) достигается абсолютный минимум.


% \addcontentsline{toc}{section}{Список литературы}
% \begin{thebibliography}{99}
%   \bibitem{RKM}\textbf{О.Б. Арушанян, С.Ф. Залеткин} Решение систем обыкновенных дифференциальных уравнений методами Рунге--Кутта.
%   \bibitem{TAC}\textbf{Д.П. Ким} Теория автоматического управления. Том 2.
%   \bibitem{OM}\textbf{Н.Л. Майорова, Д.В. Глазков} Методы оптимизации.
%   \bibitem{DEP}\textbf{А.Ф. Филиппов} Сборник задач по дифференциальным уравнениям.
% \end{thebibliography}
\end{document}