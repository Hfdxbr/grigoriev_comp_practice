\documentclass[a4paper,12pt]{article}
\usepackage[T2A]{fontenc}
\usepackage[utf8]{inputenc}
\usepackage[russian]{babel}
\usepackage[style=russian]{csquotes}
\usepackage{xcolor}
\usepackage{amsmath}
\usepackage{tikz}
\usepackage{pgfplots}
\usepackage[l3]{csvsimple}

\usepackage{cleveref}
\crefformat{equation}{(#2#1#3)}
\crefrangeformat{equation}{(#3#1#4)~-~(#5#2#6)}
\crefmultiformat{equation}{(#2#1#3)}{ и~(#2#1#3)}{, (#2#1#3)}{ и~(#2#1#3)}

\allowdisplaybreaks

\patchcmd{\thebibliography}{\section*{Список литературы}}{}{}{}

\newcommand{\UpdateMe}[1]{\textcolor{red}{#1}}
\DeclareMathOperator*{\argmax}{argmax}

\newcommand{\University}{Московский государственный университет имени М.~В.~Ломоносова}
\newcommand{\Department}{Кафедра \UpdateMe{НАЗВАНИЕ-КАФЕДРЫ}}
\newcommand{\Student}{\UpdateMe{ИМЯ-СТУДЕНТА}}
\newcommand{\GroupNum}{\UpdateMe{НОМЕР}}
\newcommand{\Seminar}{Численные методы в задачах оптимального управления}

\begin{document}
% \begin{titlepage}
%     \centering
%     {\scshape\Large \University\par}\vspace{1cm}{\scshape\large \Department\par}
%     \vfill
%     {\huge\bfseries ОТЧЕТ\par}{\Largeпо задаче практикума \enquote{\Seminar}\par}
%     \vfill
%     \hfill\begin{minipage}{0.45\linewidth}Выполнил студент гр. \GroupNum:\\\Student\end{minipage}
%     \vfill
%     {\large Москва, \the\year{}\par}
% \end{titlepage}

\section*{Задание}
\begin{gather*}
  J = \int_0^T \left(\ddot{x}^2-\dot{x}^2-x^2\right)dt \rightarrow \inf, \quad \vert\ddot{x}\vert \le 1 \\
  x(0)=x(T)=0,\quad \dot{x}(0) = 0 \\
  T\in\{0.1, 1, 10, 20\}
\end{gather*}

\section{Теория}

Введем обозначения \(x_1=x, x_2=\dot{x}, u=\ddot{x}\). Тогда задача
перепишется в виде

\begin{gather*}
  \begin{cases}
    \dot{x}_1=x_2 \\
    \dot{x}_2=u
  \end{cases} \\
  x_1(0)=0, \quad x_1(T)=0 \\
  x_2(0)=0 \\
  \int_0^T\left(u^2-x_2^2-x_1^2\right)dt \rightarrow \inf, \quad \vert u\vert \le 1
\end{gather*}

\subsection{Основные конструкции}

\subsubsection{Функция Лагранжа}

\begin{gather*}
  \hat{L}=\int_0^1L dt+l \\
  L=\lambda_0 \left(u^2-x_2^2-x_1^2\right)+p_1\left(\dot{x}_1-x_2\right)+p_2\left(\dot{x}_2-u\right)=p_1 \dot{x}_1+p_2 \dot{x}_2 - H \\
  l=\lambda_1 x_1(0)+\lambda_2 x_1(T)+\lambda_3 x_2(1)
\end{gather*}

\subsubsection{Условия оптимальности}

\begin{enumerate}
  \def\labelenumi{\arabic{enumi}.}
  \item
        Уравнения Эйлера-Лагранжа:
        \(
        \begin{cases}
          -\frac{d}{dt}L_{\dot{x}_1}+L_{x_1}=0 \\
          -\frac{d}{dt}L_{\dot{x}_2}+L_{x_2}=0
        \end{cases} \Rightarrow
        \begin{cases}
          \dot{p}_1=-2\lambda_0 x_1 \\
          \dot{p}_2=-p_1 - 2\lambda_0 x_2
        \end{cases}
        \)
  \item
        Условия
        трансверсальности:\(
        \begin{cases}
          L_{\dot{x}_1}(0)=l_{x_1}(0)  \\
          L_{\dot{x}_1}(T)=-l_{x_1}(T) \\
          L_{\dot{x}_2}(0)=l_{x_2}(0)  \\
          L_{\dot{x}_2}(T)=-l_{x_2}(T)
        \end{cases} \Rightarrow
        \begin{cases} p_1(0)=\lambda_1  \\
          p_1(T)=-\lambda_2 \\
          p_2(0)=\lambda_3  \\
          p_2(T)=0\end{cases}
        \)
  \item
        Условие оптимальности по управлению:
        \( \tilde{u}=\argmax(H)=\argmax\left(p_2u-\lambda_0 u^2\right) \Rightarrow \tilde{u}=\frac{p_2}{2\lambda_0}, \lambda_0 \neq 0 \)
  \item
        Условие невырожденности и неотрицательности:
        \( \sum_{i=0}^{3} |\lambda_i| \neq 0, \lambda_0 \ge 0 \)
\end{enumerate}

\subsection{Постановка задачи}

\subsubsection{\texorpdfstring{Разбор тривиального случая
    \((\lambda_0=0)\)}{Разбор тривиального случая (\textbackslash lambda\_0=0)}}

Пусть \(\lambda_0=0\), тогда
\(\tilde{u}=\argmax(H)=\argmax\left(p_2u\right) = \text{sign}(p_2)\) и
\(\dot{p}_1=0\).
Но тогда \(p_2=-c_{p_1} t+c_{p_2}=c_{p_1} (T - t)\),
\(\text{sign}(p_2)=\text{sign}(c_{p_1})=\text{const}\) и
\(x_2=\pm t + c_{x_2}=\pm t\). В таком случае
\(x_1=\pm \frac{1}{2} t^2+c_{x_1}=\pm \frac{1}{2} t^2\) и правое
граничное условие не удовлетворяется.

Таким образом можно заключить, что \(\lambda_0 \neq 0\).

\subsubsection{Общий случай}

В силу однородности функции Лагранжа по множителям в качестве условия
нормировки положим
\(\lambda_0=1/2 \Rightarrow u=\text{clamp}(p_2,-1,1)\). Тогда краевая
задача примет вид:

\begin{equation}
  \label{system}
  \begin{cases}
    \dot{x}_1=f_1(x_1,x_2,p_1,p_2)=x_2                    \\
    \dot{x}_2=f_2(x_1,x_2,p_1,p_2)=\text{clamp}(p_2,-1,1) \\
    \dot{p}_1=f_3(x_1,x_2,p_1,p_2)=-x_1                   \\
    \dot{p}_2=f_4(x_1,x_2,p_1,p_2)=-p_1-x_2
  \end{cases}
\end{equation}

\begin{equation}
  \label{left_cond}
  x_1(0)=0, \quad x_2(0)=0
\end{equation}

\begin{equation}
  \label{right_cond}
  x_1(T)=0, \quad p_2(T)=0
\end{equation}

\section{Методика решения}

\subsection{Численный метод решения}

\subsubsection{Метод Рунге-Кутта}

Пусть дана задача Коши

\begin{equation*}
  \begin{cases}
    \dot{x}_1=f_1(x_1,x_2,p_1,p_2) \\
    \dot{x}_2=f_2(x_1,x_2,p_1,p_2) \\
    \dot{p}_1=f_3(x_1,x_2,p_1,p_2) \\
    \dot{p}_2=f_4(x_1,x_2,p_1,p_2)
  \end{cases}
\end{equation*}

с известными в начальный момент времени значениями \(x_1^0\),
\(\ x_2^0\), \(p_1^0\) и \(p_2^0\)

Допустим уже найдены значения в точке \(t_i\): \(x_1^i\), \(\ x_2^i\),
\(p_1^i\) и \(p_2^i\). Тогда значения в точке \(t_{i+1}\) вычисляются по
схеме:

\begin{gather*}
  k_{11}=f_1(x_1^i,x_2^i,p_1^i,p_2^i) \\
  k_{21}=f_2(x_1^i,x_2^i,p_1^i,p_2^i) \\
  k_{31}=f_3(x_1^i,x_2^i,p_1^i,p_2^i) \\
  k_{41}=f_4(x_1^i,x_2^i,p_1^i,p_2^i) \\
  {}\\
  k_{12}=f_1(x_1^i + 0.5h k_{11},x_2^i + 0.5h k_{21},p_1^i + 0.5h k_{31},p_2^i + 0.5h k_{41}) \\
  k_{22}=f_2(x_1^i + 0.5h k_{11},x_2^i + 0.5h k_{21},p_1^i + 0.5h k_{31},p_2^i + 0.5h k_{41}) \\
  k_{32}=f_3(x_1^i + 0.5h k_{11},x_2^i + 0.5h k_{21},p_1^i + 0.5h k_{31},p_2^i + 0.5h k_{41}) \\
  k_{42}=f_4(x_1^i + 0.5h k_{11},x_2^i + 0.5h k_{21},p_1^i + 0.5h k_{31},p_2^i + 0.5h k_{41}) \\
  {}\\
  k_{13}=f_1(x_1^i + 0.5h k_{12},x_2^i + 0.5h k_{22},p_1^i + 0.5h k_{32},p_2^i + 0.5h k_{42}) \\
  k_{23}=f_2(x_1^i + 0.5h k_{12},x_2^i + 0.5h k_{22},p_1^i + 0.5h k_{32},p_2^i + 0.5h k_{42}) \\
  k_{33}=f_3(x_1^i + 0.5h k_{12},x_2^i + 0.5h k_{22},p_1^i + 0.5h k_{32},p_2^i + 0.5h k_{42}) \\
  k_{43}=f_4(x_1^i + 0.5h k_{12},x_2^i + 0.5h k_{22},p_1^i + 0.5h k_{32},p_2^i + 0.5h k_{42}) \\
  {}\\
  k_{14}=f_1(x_1^i + h k_{13},x_2^i + h k_{23},p_1^i + h k_{33},p_2^i + h k_{43}) \\
  k_{24}=f_2(x_1^i + h k_{13},x_2^i + h k_{23},p_1^i + h k_{33},p_2^i + h k_{43}) \\
  k_{34}=f_3(x_1^i + h k_{13},x_2^i + h k_{23},p_1^i + h k_{33},p_2^i + h k_{43}) \\
  k_{44}=f_4(x_1^i + h k_{13},x_2^i + h k_{23},p_1^i + h k_{33},p_2^i + h k_{43}) \\
  {}\\
  x_1^{i+1}=x_1^{i}+ h (k_{11}+2(k_{12}+k_{13})+k_{14}) / 6 \\
  x_2^{i+1}=x_2^{i}+ h (k_{21}+2(k_{22}+k_{23})+k_{24}) / 6 \\
  p_1^{i+1}=p_1^{i}+ h (k_{31}+2(k_{32}+k_{33})+k_{34}) / 6 \\
  p_2^{i+1}=p_2^{i}+ h (k_{41}+2(k_{42}+k_{43})+k_{44}) / 6
\end{gather*}

где \(h\) - адаптивный шаг по времени

\subsubsection{Адаптивный шаг по времени}

Пусть задан уровень погрешности \(\varepsilon\). Обозначим за \(y^i\)
значение искомой функции в текущей точке, а за \(h\) - текущий шаг по
времени.

Тогда новый шаг \(h'\) на каждой итерации будет вычисляться по следующему
алгоритму

\begin{itemize}
  \item
        Если \(u_{state}^i \neq u_{state}^{i+1}\), то \(h=h^*(y^i, u_{state}^i)\in(0, h)\) - шаг до точки разрыва
  \item
        \(err =|\text{runge\_kutta}^h(y^i)-\text{runge\_kutta}^{h/2}\left(\text{runge\_kutta}^{h/2}(y^i)\right)\)
  \item
        \(h'=h/2\), если \(err > \varepsilon\), иначе \(h'=h\)
\end{itemize}

\subsubsection{Метод стрельбы}

Для краевой задачи \cref{system} из краевых условий \cref{left_cond} имеем

\(x_1^0=0, \quad x_2^0=0\)

Два других ``начальных'' значения выберем в качестве параметров
пристрелки

\(p_1^0=a, \quad p_2^0=b\)

Тогда решая систему \cref{system} с такими начальными условиями до момента
времени \(t=T\) получим некоторое решение \(\tilde{x}_1\),
\(\tilde{x}_2\), \(\tilde{p}_1\) и \(\tilde{p}_2\).

Варьируя параметры \(a\) и \(b\) необходимо добиться выполнения краевых
условий \cref{right_cond}:

\(\tilde{x}_1(T)=x_1^N=0\) и \(\tilde{p}_2(T)=p_2^N=0\)

Будем искать параметры \(a\) и \(b\) методом простых итераций. Введём
функции \(\phi_1\) и \(\phi_2\)показывающие отклонение от заданного
граничного условия:

\begin{gather*}
  \phi_1(a,b)=x_1^N-0=0\\
  \phi_2(a,b)=p_2^N-0=0
\end{gather*}

Разложив эти функции в окрестности точки \((a^i, b^i)\) получим:

\begin{gather*}
  \phi_1(a,b)\simeq\phi_1(a^i,b^i)+\delta a\frac{\partial \phi_1}{\partial a}(a^i,b^i)+\delta b\frac{\partial \phi_1}{\partial b}(a^i,b^i)\simeq0\\ \phi_2(a,b)\simeq\phi_2(a^i,b^i)+\delta a\frac{\partial \phi_2}{\partial a}(a^i,b^i)+\delta b\frac{\partial \phi_2}{\partial b}(a^i,b^i)\simeq0
\end{gather*}

где \(\delta z=z^{i+1}-z^{i}\). В матричном виде уравнения примут вид
\begin{equation*}
  \begin{pmatrix}
    \frac{\partial \phi_1}{\partial a}(a^i,b^i) & \frac{\partial \phi_1}{\partial b}(a^i,b^i) \\
    \frac{\partial \phi_2}{\partial a}(a^i,b^i) & \frac{\partial \phi_2}{\partial b}(a^i,b^i) \\
  \end{pmatrix} \cdot
  \begin{pmatrix}
    \delta a \\
    \delta b
  \end{pmatrix} =W\cdot
  \begin{pmatrix}
    \delta a \\
    \delta b
  \end{pmatrix} = -
  \begin{pmatrix}
    \phi_1(a^i,b^i) \\
    \phi_2(a^i,b^i)
  \end{pmatrix}
\end{equation*}

Тогда итерационный процесс может быть выражен формулой

\begin{equation*}
  \begin{pmatrix}
    a^{i+1} \\
    b^{i+1}
  \end{pmatrix}=
  \begin{pmatrix}
    a^{i} \\
    b^{i}
  \end{pmatrix}-W^{-1}\cdot
  \begin{pmatrix}
    \phi_1(a^i,b^i) \\
    \phi_2(a^i,b^i)
  \end{pmatrix}
\end{equation*}

\subsection{Аналитическое решение}

Из-за наложенного ограничения на управление явно построить аналитическое
решение не представляется возможным.

\section{Результаты}
Представленные ниже результаты получены при уровне погрешности \(\varepsilon~=~10^{-10}\).
\subsection*{Случай \(T=0.1\)}
\begin{tikzpicture}
  \begin{axis}[width=15cm, height=8cm, legend pos=outer north east, grid = major, grid style={dashed, gray!30},x tick label style={/pgf/number format/fixed, rotate=45}]
    \addplot [thick, mark = none, red]table [x=t, y=x1, col sep=comma] {data/T0.1.csv};
    \addlegendentry{$x_1$}
    \addplot [thick, mark = none, blue]table [x=t, y=x2, col sep=comma] {data/T0.1.csv};
    \addlegendentry{$x_2$}
    \addplot [thick, mark = none, green]table [x=t, y=p1, col sep=comma] {data/T0.1.csv};
    \addlegendentry{$p_1$}
    \addplot [thick, mark = none, black]table [x=t, y=p2, col sep=comma] {data/T0.1.csv};
    \addlegendentry{$p_2$}
  \end{axis}
\end{tikzpicture}

\subsection*{Случай \(T=1.0\)}
\begin{tikzpicture}
  \begin{axis}[width=15cm, height=8cm, legend pos=outer north east, grid = major, grid style={dashed, gray!30}]
    \addplot [thick, mark = none, red]table [x=t, y=x1, col sep=comma] {data/T1.0.csv};
    \addlegendentry{$x_1$}
    \addplot [thick, mark = none, blue]table [x=t, y=x2, col sep=comma] {data/T1.0.csv};
    \addlegendentry{$x_2$}
    \addplot [thick, mark = none, green]table [x=t, y=p1, col sep=comma] {data/T1.0.csv};
    \addlegendentry{$p_1$}
    \addplot [thick, mark = none, black]table [x=t, y=p2, col sep=comma] {data/T1.0.csv};
    \addlegendentry{$p_2$}
  \end{axis}
\end{tikzpicture}

\subsection*{Случай \(T=10.0\)}
\begin{tikzpicture}
  \begin{axis}[width=15cm, height=8cm, legend pos=outer north east, grid = major, grid style={dashed, gray!30}]
    \addplot [thick, mark = none, red]table [x=t, y=x1, col sep=comma] {data/T10.0.csv};
    \addlegendentry{$x_1$}
    \addplot [thick, mark = none, blue]table [x=t, y=x2, col sep=comma] {data/T10.0.csv};
    \addlegendentry{$x_2$}
    \addplot [thick, mark = none, green]table [x=t, y=p1, col sep=comma] {data/T10.0.csv};
    \addlegendentry{$p_1$}
    \addplot [thick, mark = none, black]table [x=t, y=p2, col sep=comma] {data/T10.0.csv};
    \addlegendentry{$p_2$}
  \end{axis}
\end{tikzpicture}

\subsection*{Случай \(T=20.0\)}
\begin{tikzpicture}
  \begin{axis}[width=15cm, height=8cm, legend pos=outer north east, grid = major, grid style={dashed, gray!30}]
    \addplot [thick, mark = none, red]table [x=t, y=x1, col sep=comma] {data/T20.0.csv};
    \addlegendentry{$x_1$}
    \addplot [thick, mark = none, blue]table [x=t, y=x2, col sep=comma] {data/T20.0.csv};
    \addlegendentry{$x_2$}
    \addplot [thick, mark = none, green]table [x=t, y=p1, col sep=comma] {data/T20.0.csv};
    \addlegendentry{$p_1$}
    \addplot [thick, mark = none, black]table [x=t, y=p2, col sep=comma] {data/T20.0.csv};
    \addlegendentry{$p_2$}
  \end{axis}
\end{tikzpicture}

\subsection*{Сводная таблица}
\csvautotabular{data/stats.csv}

\subsection*{Анализ результатов}
Для случаев \(T=0.1\) и \(T=1.0\) найденные значения пристрелки порядка уровня погрешности, то есть их можно принять равными \(0\). Данный факт подтверждается теоретически. Действительно из ограничения управления следует
\begin{gather*}
  \vert \ddot{x(t)} \vert \le s, \quad s \in [0, 1] \\
  \Rightarrow \vert \dot{x}(t) \vert \le s t, \text{ т.к. } \dot{x}(0) = 0 \\
  \Rightarrow \vert x(t) \vert \le \frac{1}{2} s t^2, \text{ т.к. } x(0) = 0
\end{gather*}
Но тогда \(L \ge s \left(1 - t - \frac{1}{2} t^2\right)\) и \(J \ge s \int_{0}^{T} \left(1 - t - \frac{1}{2} t^2\right)dt\). Соответственно минимум достигается при \(s=0\). Следовательно \(x\equiv0\), но тогда из системы \cref{system} \(p_2\equiv0\) и \(p_1\equiv0\).

\section*{Анализ оптимальности}
Для исследования оптимальности найденных экстремалей проверим условия второго порядка:
\begin{itemize}
  \item \textbf{Условие Лежандра}. Так как \(L_{uu}=2 > 0\), то выполняется усиленное условие Лежандра, и необходимое условие минимума выполнено.
  \item \textbf{Условие Якоби}. Точка \(\tau\) является сопряжённой, если существует нетривиальное решение следующей краевой задачи:
        \begin{gather*}
          \begin{cases}
            \dot{\delta x}_1=\delta x_2 \\
            \dot{\delta x}_2=q_2        \\
            \dot{q}_1=-\delta x_1       \\
            \dot{q}_2=-q_1 -\delta x_2
          \end{cases}\\
          \delta x_1(\tau)=\delta x_2(\tau)=0 \\
          \delta x_1(0) = \delta x_2(0)=0 \\
          \delta x_1(T) = q_2(T)=0
        \end{gather*}

        Аналитическое решение данной системы показывает, что решение может быть только тривиальным. Таким образом сопряженные точки отсутствуют и выполняется усиленное условие Якоби.
  \item \textbf{Условие Вейерштрасса}. Так как \(L_{uu}=2\) положительно определена, то выполняется условие квазирегулярности (усиленное условие Вейерштрасса).
  \item \textbf{Исследование второй вариации функции Лагранжа}. Функция Лагранжа и её вторая вариация для данной задачи имеют вид:
        \begin{equation*}
          \hat{L}=\int_0^TL dt+l,\quad \hat{L}_{\xi\xi}=\int_0^T(\delta u)^2 dt
        \end{equation*}
        Достаточным условием второго порядка является положительная определённость \(\hat{L}_{\xi\xi}\) на допустимых вариациях:
        \begin{equation*}
          \hat{L}_{\xi\xi}(\delta\xi,\delta\xi) \ge \varepsilon \phi(\delta\xi), \quad \varepsilon > 0
        \end{equation*}
        В качестве \(\phi(\delta\xi)\) можно взять саму функцию \(\hat{L}\). Условие выполняется.
\end{itemize}
Так как задача линейно квадратичная, в \(\tilde{x}\) достигается абсолютный минимум.

% \addcontentsline{toc}{section}{Список литературы}
% \begin{thebibliography}{99}
%   \bibitem{RKM}\textbf{О.Б. Арушанян, С.Ф. Залеткин} Решение систем обыкновенных дифференциальных уравнений методами Рунге--Кутта.
%   \bibitem{TAC}\textbf{Д.П. Ким} Теория автоматического управления. Том 2.
%   \bibitem{OM}\textbf{Н.Л. Майорова, Д.В. Глазков} Методы оптимизации.
%   \bibitem{DEP}\textbf{А.Ф. Филиппов} Сборник задач по дифференциальным уравнениям.
% \end{thebibliography}
\end{document}